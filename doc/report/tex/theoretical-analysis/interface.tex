% ================================================================================================================================== %
% ---------------------------------------------------------- Interface ------------------------------------------------------------- %
% ================================================================================================================================== %

\section{Interfejs użytkownika}

Ostatnim elementem projektu jest zaimplementowanie wygodnego interfejsu użytkownika. Powinien on umożliwiać niezależną aktywację każdego z~efektów oraz pozwalać na regulowanie ich parametrów. Układ FPGA udostępniać będzie cztery wejścia cyfrowe (przyciski). Zostaną one podłączone (za pośrednictwem przerzutników) do wejść \verb|enable| poszczególnych bloków przetwarzających. Aktywacja efektu możliwa będzie dzięki zmianie stanu przełącznika. Kontrola parametrów bloków przetwarzana zostanie zrealizowana za pomocą potencjometrów. Ich interfejs może zostać stworzony na dwa sposoby. Ostateczny wybór zostanie podjęty po głęszbym przeanalizowaniu zagadnienia. 

Pierwszym pomysłem jest wykorzystanie modułu XADC od Xilinx. Pozwoliłoby to na podłączenie wszystkich potencjometrów do układu FPGA z~wykorzystaniem multipleksera analogowego (np. CD74HC4067), którego wejście przełączane byłoby z~pewną częstotliwością przez układ sterujący. Wymagałoby to jednak wykorzystanie gotowego bloku IP oraz stworzenia odpowiedniego interfejsu od strony aplikacji.

Drugim pomysłem jest wdrożenie dodatkowego modułu UART, który komunikowałby się z~zewnętrznym mikrokontrolerem realizującym pomiar napięć na potencjometrach poprzez wbudowane kanały przetwornika A/C. W~takim przypadku układ FPGA odpytywałby podrzędny względem niego mikrokontroler w~sposób cykliczny, pozyskując cyfrowe wartości orientacji potencjometrów. Niezależnie od wyboru metody cyfrowe wartości pomiarów zostaną podłączone poprzez przerzutniki do wejśc konfiguracyjnych odpowiendich bloków przetwarzających sygnały.
