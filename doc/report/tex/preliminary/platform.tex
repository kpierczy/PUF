% ================================================================================================================================== %
% ----------------------------------------------------------- Platform ------------------------------------------------------------- %
% ================================================================================================================================== %

\section{Wybór platformy}

Ostateczny wybór platformy zostanie dokonany po stworzeniu fundamentów projektu pozwalających oszacować wymagania urządzenia dotyczące zasobów układu FPGA. Na ten moment potencjalny wybór został zawężony do trzech zestawów ewaluacyjnych. Pierwszy z~nich to \textbf{Digilent Cmod A7} wyposażony w~moduł XC7A35T-1CPG236C z~rodziny Artix-7. Niewielkie rozmiary oraz cena nieprzekraczająca 400zł są największymi zaletami tego wariantu. Został on wyposażony w~8-bitową pamięć SRAM o pojemności 512KB oraz pamięć szeregową Quad-SPI wielkości 4MB. Druga z~rozważanych platform to \textbf{Digilent Arty S7}. Jest to układ bardziej rozbudowany od poprzedniego. Posiada on na pokładzie moduł XC7S50-1CSGA324C (Spartan-7) zawierający ponad $50\%$ więcej bloków logicznych. Sama płytka oferuje ponadto kilka diod LED, przełączników mono- i~bistabilnych oraz cztery złącza Pmod. Cena tej platformy jest o~około $18\%$ wyższa, co oznacza wysoki stosunek zasobów do ceny, jednak liczyć się trzeba ze znacznie większymi wymiarami płytki.

Trzecim wariantem jest z~kolei zestaw \textbf{Digilent Cora Z7}. Jest to układ najuboższy w~bloki logiczne a~przy tym wyceniany na podobnym poziomie co Arty S7. Posiada on jednak układ XC7Z007S-1CLG400C z~rodziny Zynq co oznacza, że poza modułem FPGA zawiera on również jednordzeniowy procesor w~architekturze ARM Cortex-A9. Zestaw ten brany jest pod uwagę jedynie ze względu na prywatną sympatię autora do mikrokontrolerów bazujących na architekturze Cortex-M i~wynikającej z~niej chęci zapoznania się z~architekturą aplikacyjną rodziny ARM. Wariant ten zostanie wybrany wówacz, jeżeli zasoby zawartego w~nim modułu FPGA okażą się wystarczające do zaimplementowania tworzonego rozwiązania.
