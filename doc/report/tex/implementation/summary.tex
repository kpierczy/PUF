\section{Podsumowanie}

Zrealizowany projekt stanowił solidny wstęp do projektowania układów cyfrowych z~wykorzystaniem języka VHDL\footnote{W~niektórych kręgach można by nawet określić, że było to istny \textit{skok na głęboką wodę}.}. Pozwolił on nie tylko zapoznać się ze składnią języka w~stopniu umożliwiającym jego (stosunkowo) swobodne wykorzystywanie, ale także wdrożyć się w~sposób myślenia w~kategoriach konfiguracji zasobów sprzętowych tak różnego od tego, co znane jest z~tworzenia oprogramowania. Dodatkowym benefitem płynącym z~jego realizacji była możliwość poznania platformy projektowej w~postaci oprogramowania \textit{Vivado}, które umożliwi w~przyszłości znacznie szybsze zainicjalizowanie właściwej części prac nad nowymi projektami\footnote{Zapoznanie się z~rzeczonym oprogramowaniem utwierdziło także autora w~przekonaniu, że zintegrowane środowiska projektowe dostarczane przez dużych producentów nie są tym, co misie lubią najbardziej. W~przypadku przyszłych projektów duży nacisk zostanie położony na wykorzystanie skryptów języka \textit{Tcl}, które umożliwią jeszcze większą separację projektu od tego typu rozwiązań na rzecz środowiska stworzonego według własnych gustów i~przyzwyczajeń.}.

Ilość pracy włożonej w~projekt sprawiła, że stworzony w~jego ramach kod postanowiono doprowadzić do postaci, w~której będzie mógł on zostać wykorzystany z~autentycznym sprzętem gitarowym. Wymagać to będzie stworzenia odpowiedniej platformy sprzętowej. Zadanie to obejmuje określenie wykorzystanych komponentów (ze szczególnym uwzględnieniem elementów analogowych) oraz (potencjalnie) zaprojektowanie własnego obwodu drukowanego. Na dzień dzisiejszy prace te zostają odłożone w~czasie do momentu ukończenia projektów realizowany w~ramach jakże służnie mijającego semestru.
